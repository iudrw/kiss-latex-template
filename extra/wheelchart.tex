\newcommand{\loadedWheelchart}{}

%
% https://github.com/sylhare/Latex/blob/master/src/wheelchart.tex
%

%%%%%%%%%%%%%%%%%%%%%%%%%%%%%%%%%%%%%%%%%%%%%%
% Adapted and modified from this posts:
% http://tex.stackexchange.com/questions/17898/how-can-i-produce-a-ring-or-wheel-chart-like-that-on-page-88-of-the-pgf-manu
% http://tex.stackexchange.com/questions/82727/create-a-ring-diagram-in-tex/82729#82729

%%%%%%%%%%%%%%%%%%%%%%%%%%%%%%%%%%%%%%%%%%%%%%

\usepackage{tikz}

% The main wheelchart macro

\newcommand{\wheelchart}[6]{
	
    % Calculate total
    \pgfmathsetmacro{\totalnum}{0}
    \foreach \value/\colour/\name in {#1} {
        \pgfmathparse{\value+\totalnum}
        \global\let\totalnum=\pgfmathresult
    }

    \begin{tikzpicture}
      % The text in the center of the wheel
      \node[align=center,text width=2*#3]{#2};

      % Calculate the thickness and the middle line of the wheel
      \pgfmathsetmacro{\wheelwidth}{#4-#3}
      \pgfmathsetmacro{\midradius}{(#4+#3)/2}

      % Rotate so we start from the top
      \begin{scope}[line width=\wheelwidth,rotate=90]

      % Loop through each value set. \cumnum keeps track of where we are in the wheel
      \pgfmathsetmacro{\cumnum}{0}
      \foreach \value/\colour/\name in {#1} {
            \pgfmathsetmacro{\newcumnum}{\cumnum + \value/\totalnum*360}

            % Calculate the percent value
            \pgfmathsetmacro{\percentage}{\value/\totalnum*100}
            
            % Calculate the mid angle of the colour segments to place the labels
            \pgfmathsetmacro{\midangle}{-(\cumnum+\newcumnum)/2}

			% This is necessary for the labels to align nicely
			\pgfmathparse{
				(-\midangle<180?"west":"east")
			} \edef\textanchor{\pgfmathresult}
			\pgfmathsetmacro\labelshiftdir{1-2*(-\midangle>180)}

            % Draw the color segments. Somehow, the \midrow units got lost, so we add 'pt' at the end. Not nice...
            \draw[\colour] (-\cumnum:\midradius pt) arc (-\cumnum:-(\newcumnum):\midradius pt);

            % Draw the data labels
            \node at (\midangle:#4 + 1ex) [inner sep=0pt, outer sep=0pt, ,anchor=\textanchor]{\name};


			\ifx&#5&
				%Do Nothing if empty
			\else
				% The 'spokes' ()gray bars)
				\foreach \i in {0,...,\value} {
					\draw [#5,thin] (-\cumnum-\i/\totalnum*360:#3) -- (-\cumnum-\i/\totalnum*360:#4);
				}	
			\fi
            
            % Set the old cumulated angle to the new value
            \global\let\cumnum=\newcumnum
        }

      \end{scope}
      
      \ifx&#6&
          %Do Nothing if empty
      \else
	      %Draw the inner and outer circles
	      \draw[#6] (0,0) circle (#4) circle (#3);
      \fi
      
    \end{tikzpicture}
}

% Usage: 
% \wheelchart {<value1>/<colour1>/<label1>, ...} {Name} {innerRadius} {outerRadius} {innerStrokesColor} {outLinesColor}

% \wheelchart{20/green/good,  10/yellow/medium, 9/red/bad, 5/white/neutral}{Ratings}{1.8cm}{2.2cm}{gray}{black}
