% A simple file with package imports categorized

% Look up any packages with:
% https://ctan.org/pkg/PACKAGE

% ========================================================
%                    🖨️ PDF OPTIONS
% ========================================================

\iffalse % THIS IS A MULTI-LINE COMMENT

\usepackage[3-4]{pagesel}
    % Selects single pages, ranges of pages, odd pages or even pages for output.

\fi

% ========================================================
%                    📐 MARGINS
%                    SPACING, PARAGRAPHS
% ========================================================

\usepackage[margin=1.0in]{geometry}
    % The package provides an easy and flexible user interface to customize page layout.

% ========================================================
%                    📚 LANGUAGE
%                    BIBLIOGRAPHY
% ========================================================

\usepackage[main=english,portuguese]{babel}
    % This package manages culturally-determined typographical (and other) rules for a wide range of languages. I set the language to danish using babel, to ensure proper quoting and hyphenation.

\usepackage{csquotes}
    % The csquotes package supplies typographically correct quotations instead of backticks and apostrophes.

\usepackage[style=numeric,backend=biber]{biblatex}
    % BibLATEX is a complete reimplementation of the bibliographic facilities provided by LATEX.
    
    \DeclareLanguageMapping{english}{english-apa}
        % This command maps a babel/polyglossia language identifier to an lbx file
    
    % - - - - - 📜 BIBLIOGRAPHY FILES - - - - -
    \addbibresource{extra/latex/examples.bib}

\usepackage{hyperref}
    % Extends the functionality of all the LATEX cross-referencing commands (including the table of contents, bibliographies etc) to produce \special commands which a driver can turn into hypertext links.

    \hypersetup{
        colorlinks, linkcolor=black, anchorcolor=black, citecolor=black,
        filecolor=black, menucolor=black, urlcolor=black, urlcolor=black,
    }

% ========================================================
%                    ✍🏽 FONTS
%                    MATH
% ========================================================

\usepackage{color}
    % The color package provides both foreground and background colour management.

    % - - - - - 🌈 COLORS - - - - -
    
    \definecolor{mygreen}{rgb}{0,0.6,0}

\usepackage{amsmath}
    % American Mathematical Society Math Extensions

\usepackage{fontspec}
    % Fontspec is a package for XELATEX and LuaLATEX compilers. It provides an automatic and unified interface to feature-rich AAT and OpenType fonts through the NFSS in LATEX running on XETEX or LuaTEX engines.
    
    \setmainfont{Calibri}

% ========================================================
%                    🖼️ FIGURES, CHARTS
%                    TABLES, CODE BLOCKS
% ========================================================

\usepackage{float}
    % Improves the interface for defining floating objects such as figures and tables.

\usepackage{graphicx}
    % The package for inclusion of graphics in LATEX documents.

    % - - - - - Setting figures location/path - - - - -
    \graphicspath{{figures/}}

\usepackage[export]{adjustbox}
    % The package provides several macros to adjust boxed content (e.g. vertical alignment of figures).

\usepackage{array}
    % An extended implementation of the array and tabular environments which extends the options for column formats, and provides “programmable” format specifications.

% ========================================================
%                    🔌 EXTRA
% ========================================================

%\newcommand{\loadedFonts}{}

%\usepackage{fontspec}

% - - - - - 🅰️ FONTS - - - - -

\newfontfamily\fontawesome{Font Awesome 6 Free-Regular-400}[
    Path = {extra/fonts/fontawesome-free-6.1.0/otfs/}, Extension = {.otf},
    FontFace = {m}{solid}{Font Awesome 6 Free-Solid-900},
    FontFace = {m}{brands}{Font Awesome 6 Brands-Regular-400}
]

\newfontfamily\fontbadcomic{BadComic-MVw8P}[
    Path = {extra/fonts/bad-comic-font/}, Extension = {.ttf},
    ItalicFont = {BadComicItalic-K72RX}
]

%\input{extra/wheelchart}
