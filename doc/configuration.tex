% A simple file with package imports categorized

% Look up any packages with:
% https://ctan.org/pkg/PACKAGE

% ========================================================
%                    🖨️ PDF OPTIONS
% ========================================================

\iffalse % THIS IS A MULTI-LINE COMMENT

\usepackage[3-4]{pagesel}
    % Selects single pages, ranges of pages, odd pages or even pages for output.

\fi

% ========================================================
%                    📐 MARGINS
%                    SPACING, PARAGRAPHS
% ========================================================

\usepackage[margin=1.0in]{geometry}
    % The package provides an easy and flexible user interface to customize page layout.

% ========================================================
%                    📚 LANGUAGE
%                    BIBLIOGRAPHY
% ========================================================

\usepackage[main=english,portuguese]{babel}
    % This package manages culturally-determined typographical (and other) rules for a wide range of languages. I set the language to danish using babel, to ensure proper quoting and hyphenation.

\usepackage{csquotes}
    % The csquotes package supplies typographically correct quotations instead of backticks and apostrophes.

\usepackage[style=numeric,backend=biber]{biblatex}
    % BibLATEX is a complete reimplementation of the bibliographic facilities provided by LATEX.
    
    \DeclareLanguageMapping{english}{english-apa}
        % This command maps a babel/polyglossia language identifier to an lbx file
    
    % - - - - - 📜 BIBLIOGRAPHY FILES - - - - -
    \addbibresource{extra/latex/examples.bib}

\usepackage{hyperref}
    % Extends the functionality of all the LATEX cross-referencing commands (including the table of contents, bibliographies etc) to produce \special commands which a driver can turn into hypertext links.

    \hypersetup{
        colorlinks, linkcolor=black, anchorcolor=black, citecolor=black,
        filecolor=black, menucolor=black, urlcolor=black, urlcolor=black,
    }

% ========================================================
%                    ✍🏽 FONTS
%                    MATH
% ========================================================

\usepackage{color}
    % The color package provides both foreground and background colour management.

    % - - - - - 🌈 COLORS - - - - -
    
    \definecolor{mygreen}{rgb}{0,0.6,0}

\usepackage{amsmath}
    % American Mathematical Society Math Extensions

\usepackage{fontspec}
    % Fontspec is a package for XELATEX and LuaLATEX compilers. It provides an automatic and unified interface to feature-rich AAT and OpenType fonts through the NFSS in LATEX running on XETEX or LuaTEX engines.
    
    \setmainfont{Calibri}

% ========================================================
%                    🖼️ FIGURES, CHARTS
%                    TABLES, CODE BLOCKS
% ========================================================

\usepackage{float}
    % Improves the interface for defining floating objects such as figures and tables.

\usepackage{graphicx}
    % The package for inclusion of graphics in LATEX documents.

    % - - - - - Setting figures location/path - - - - -
    \graphicspath{{figures/}}

\usepackage[export]{adjustbox}
    % The package provides several macros to adjust boxed content (e.g. vertical alignment of figures).

\usepackage{array}
    % An extended implementation of the array and tabular environments which extends the options for column formats, and provides “programmable” format specifications.

% ========================================================
%                    🔌 EXTRA
% ========================================================

%%\usepackage{fontspec}

% - - - - - 🅰️ FONTS - - - - -

\newfontfamily\fontawesome{Font Awesome 6 Free-Regular-400}[
    Path = {extra/fonts/fontawesome-free-6.1.0/otfs/}, Extension = {.otf},
    FontFace = {m}{solid}{Font Awesome 6 Free-Solid-900},
    FontFace = {m}{brands}{Font Awesome 6 Brands-Regular-400}
]

\newfontfamily\fontbadcomic{BadComic-MVw8P}[
    Path = {extra/fonts/bad-comic-font/}, Extension = {.ttf},
    ItalicFont = {BadComicItalic-K72RX}
]

%\newcommand{\loadedWheelchart}{}

%
% https://github.com/sylhare/Latex/blob/master/src/wheelchart.tex
%

%%%%%%%%%%%%%%%%%%%%%%%%%%%%%%%%%%%%%%%%%%%%%%
% Adapted and modified from this posts:
% http://tex.stackexchange.com/questions/17898/how-can-i-produce-a-ring-or-wheel-chart-like-that-on-page-88-of-the-pgf-manu
% http://tex.stackexchange.com/questions/82727/create-a-ring-diagram-in-tex/82729#82729

%%%%%%%%%%%%%%%%%%%%%%%%%%%%%%%%%%%%%%%%%%%%%%

\usepackage{tikz}

% The main wheelchart macro

\newcommand{\wheelchart}[6]{
	
    % Calculate total
    \pgfmathsetmacro{\totalnum}{0}
    \foreach \value/\colour/\name in {#1} {
        \pgfmathparse{\value+\totalnum}
        \global\let\totalnum=\pgfmathresult
    }

    \begin{tikzpicture}
      % The text in the center of the wheel
      \node[align=center,text width=2*#3]{#2};

      % Calculate the thickness and the middle line of the wheel
      \pgfmathsetmacro{\wheelwidth}{#4-#3}
      \pgfmathsetmacro{\midradius}{(#4+#3)/2}

      % Rotate so we start from the top
      \begin{scope}[line width=\wheelwidth,rotate=90]

      % Loop through each value set. \cumnum keeps track of where we are in the wheel
      \pgfmathsetmacro{\cumnum}{0}
      \foreach \value/\colour/\name in {#1} {
            \pgfmathsetmacro{\newcumnum}{\cumnum + \value/\totalnum*360}

            % Calculate the percent value
            \pgfmathsetmacro{\percentage}{\value/\totalnum*100}
            
            % Calculate the mid angle of the colour segments to place the labels
            \pgfmathsetmacro{\midangle}{-(\cumnum+\newcumnum)/2}

			% This is necessary for the labels to align nicely
			\pgfmathparse{
				(-\midangle<180?"west":"east")
			} \edef\textanchor{\pgfmathresult}
			\pgfmathsetmacro\labelshiftdir{1-2*(-\midangle>180)}

            % Draw the color segments. Somehow, the \midrow units got lost, so we add 'pt' at the end. Not nice...
            \draw[\colour] (-\cumnum:\midradius pt) arc (-\cumnum:-(\newcumnum):\midradius pt);

            % Draw the data labels
            \node at (\midangle:#4 + 1ex) [inner sep=0pt, outer sep=0pt, ,anchor=\textanchor]{\name};


			\ifx&#5&
				%Do Nothing if empty
			\else
				% The 'spokes' ()gray bars)
				\foreach \i in {0,...,\value} {
					\draw [#5,thin] (-\cumnum-\i/\totalnum*360:#3) -- (-\cumnum-\i/\totalnum*360:#4);
				}	
			\fi
            
            % Set the old cumulated angle to the new value
            \global\let\cumnum=\newcumnum
        }

      \end{scope}
      
      \ifx&#6&
          %Do Nothing if empty
      \else
	      %Draw the inner and outer circles
	      \draw[#6] (0,0) circle (#4) circle (#3);
      \fi
      
    \end{tikzpicture}
}

% Usage: 
% \wheelchart {<value1>/<colour1>/<label1>, ...} {Name} {innerRadius} {outerRadius} {innerStrokesColor} {outLinesColor}

% \wheelchart{20/green/good,  10/yellow/medium, 9/red/bad, 5/white/neutral}{Ratings}{1.8cm}{2.2cm}{gray}{black}

